\documentclass[a4paper,11pt]{article}
\usepackage[utf8]{inputenc}
\usepackage{times}
\usepackage{algorithm}
\usepackage{algpseudocode}

\title{Report for assignment 4}
\author{Vasudha Todi (14EC10059)}

\begin{document}

\maketitle


\paragraph{Last Man Standing}
\begin{enumerate}
 \item \textbf{To determine the elimination order and hence the survivor according to the given scheme}\\
The elimination order is determined using a recursive function. A relation is derived between V(n) and V(n/2) and the survivor is obtained using V(1).\\
$V(n) = V(n/2)*2^i - d$, where $d = d_{prev} + (-1)^n*2^{i-1}$ and i is the iteration number.

  \begin{algorithm}
    \caption{Last Man Standing}
    \begin{algorithmic}[1]
    \State{survivor}{$(n,j,d)$}
    \State $d =  d +/-  2^{j-1}$               //sign according to previous n
    \If{$n = 1$}
        \State return $(2^j - d)$
    \EndIf
    \For{$(i = 1; i <= n/2; i++)$}
        \State print $(2*i)*2^j - d$
    \EndFor
    \State if n is odd eliminate first element also
    \State return survivor(n/2,j+1,d)
    \end{algorithmic}
  \end{algorithm}
  
  The survivor can also be obtained from the formula :\\
  $V(2^m +l) = 2*l + 1$, for $m>=0$ and $0<=l<2*m$
 
 \item \textbf{Results :}
 
 Value of n : 11\\
 Output :\\
 The elimination order is : 2, 4, 6, 8, 10, 1, 5, 9, 3, 11\\
 The survivor is : 7\\
 The survivor from the formula is : 7

 \item \textbf{Analytical check of the formula to obtain the survivor position}
 If we tabulate the value of n and V(n),\\\\
 $n\;\;   =    \; 1, \; 2, \; 3, \; 4, \; 5, \; 6, \; 7, \; 8, \; 9, \; 10\\
  V(n) =  1, \; 1, \; 3, \; 1, \; 3, \; 5, \; 7, \; 1, \; 3, \; 5$\\
 So, V(n) is an increasing odd sequence starting with V(n)=1 whenever n is a power of 2. Thus if we choose m and l such that $n = 2^m + l$ such that $0<=l<2*m$, then $V(n) = 2*l+1$. Or,\\
 $V(2^m +l) = 2*l + 1$, for $m>=0$ and $0<=l<2*m$\\
 Hence proved.
 
 \textbf{END}
\end{enumerate}

\end{document}
