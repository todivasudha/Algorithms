\documentclass[a4paper,11pt]{article}
\usepackage[utf8]{inputenc}
\usepackage{times}
\usepackage{algorithm}
\usepackage{algpseudocode}

\title{Report for assignment 7}
\author{Vasudha Todi (14EC10059)}

\begin{document}

\maketitle


\paragraph{Red Black Trees}
\begin{enumerate}
 \item \textbf{Insertion of elements in a red-black tee}
Insertion is done by performing BST insert followed by steps to maintain the red-black properties. It involves rotations and recoloring.

  \begin{algorithm}
    \caption{Inserting in a red black tree}
    \begin{algorithmic}[1]
    \State{rbInsert}{$(T,n)$}
    \State bstInsert(T, n)
    \State n.color = red
    \State x=n
    \While{$x!=NULL \:and\: x.parent.color = red$}
        \State Mark the uncle node (parent.parent.otherchild) as u
        \If{u.color = red}
            \State u.color = p.color = black
            \State p.parent.color = red
            \State x = p.parent
        \EndIf
        \If{$u.color = black \:and\: p = p.parent.left$}
            \If{x = p.right}
                \State leftRotate about p
            \EndIf
            \State rightRotate about p.parent
            \State recolor
            \State return T
        \EndIf
        \If{$u.color = black \:and\: p = p.parent.right$}
            \State Do the same thing for the opposite side
            \State return T
        \EndIf
    \EndWhile
    return T
    \end{algorithmic}
  \end{algorithm}


 \item \textbf{Deletion in a red-black tree}
Deletion is carried out for a leaf node or the 2nd last node. Here the operation is carried out according to the sibling. This is a complex process.


  \begin{algorithm}
    \caption{Deleting in a red-black tree}
    \begin{algorithmic}[1]
    \State{rbDelete}{$(T,n)$}
    \If{n.color = red}
        \State Simply remove the node
    \Else
        \State x=n
        \While{$x!=root \:and\: x!=NULL$}
            \State s = x.parent.otherchild
            \If{s has a red child and s = s.parent.right}
                \State leftRotate s.parent
                \State recolor
                \State x = NULL
            \ElsIf{s has a red child and s = s.parent.left}
                \State rightRotate s.parent
                \State recolor
                \State x = NULL
            \Else
                \State s.color = red
                \State x = x.parent
            \EndIf
        \EndWhile
    \EndIf
    \State free(n)
    \State return T
    \end{algorithmic}
  \end{algorithm}
  
  \item \textbf{Complexity of insert and delete}
 Time for each rotation : O(1)\\
 Time for recoloring : O(1)\\
 Time to traverse the tree : O(logn)\\
 In the worst case scenario, the insertion or deletion involves traversing the whole height of the tree and at each step rotations and recoloring are performed.\\
 $T(n) = O(logn) + O(1)
       = O(logn)$
       
 \item \textbf{Process Management}
Two functions, process creator and scheduler are used. There are M processes and a maximum of N node in the tree. If t is time for a process and i its priority, then number of time insertion or deletion is carried out for it = t/(i*50). So, for each process the time taken will be t/(i*50)*O(logN).So, for M process it is (t/(i*50))*M*O(logN).

\end{enumerate}

\end{document}
