\documentclass[a4paper,11pt]{article}
\usepackage[utf8]{inputenc}
\usepackage{times}
\usepackage{algorithm}
\usepackage{algpseudocode}

\title{Report for assignment 9}
\author{Vasudha Todi (14EC10059)}

\begin{document}

\maketitle

\paragraph{Fibonacci heaps}
\begin{enumerate}
 \item \textbf{Insertion in heap}
The element is just added to the root list.\\
  \begin{algorithm}
    \caption{Insertion}
    \begin{algorithmic}[1]
    \State{insertHeap}{$(H,a,n)$}
    \State create new node with the element to be inserted
    \State insert the element in the root list
    \If{$a < H.a || H == NULL$}
        \State H = a
    \EndIf
    \State n++
    \State return H
    \end{algorithmic}
  \end{algorithm}
The actual as well as amortized cost is O(1).

\item \textbf{Union of two heaps}\\
In this just the root lists of both the heaps are concatenated.
  \begin{algorithm}
    \caption{Union}
    \begin{algorithmic}[1]
    \State{unionHeap}{$(H1,H2,n1,n2)$}
    \State H = makeHeap()
    \If{H1 = NULL}
        \State return H2
    \EndIf 
    \If{H2 = NULL}
        \State return H1
    \EndIf
    \State H = H1
    \State concatenate H2 to the root list of H
    \If{$H1.a > H2.a$}
        \State H = H2
    \EndIf
    \State n = n1 + n2
    \algstore{myalg}
\end{algorithmic}
\end{algorithm}

\begin{algorithm}                     
\begin{algorithmic} [1]
\algrestore{myalg}
    \State return H
    \end{algorithmic}
  \end{algorithm}

This operation is also performed in O(1) time.

 \item \textbf{Extracting min node in heap}
In this the min is extracted along with consolidation.

  \begin{algorithm}
    \caption{Extract Min}
    \begin{algorithmic}[1]
    \State{extractMinHeap}{$(H,n)$}
    \If{$H != NULL$}
        \For{each child x of H}
            \State add x to the root list of H
            \State p[x] = NULL
        \EndFor
        \State remove the min node from the root list of H
        \State H = right[H]
        \State CONSOLIDATE(H)
        \State n--
    \EndIf
    \State return H
    \end{algorithmic}
  \end{algorithm}

The consolidate function iterates through the root list and if two node have the same degree then it merges them.\\
The degree of the any root node is bounded by log(n). Thus the amortized cost for extract min is O(log(n)).

 \item \textbf{Decreasing key in heap}
Decrease key cuts the decreased node if lesser than its parent and include it in the root list. Then cascading cut is performed.

  \begin{algorithm}
    \caption{Decrease key}
    \begin{algorithmic}[1]
    \State{decreaseKeyHeap}{$(H,x,k)$}
    \If{$x.a < k$}
        \State Print("greater key")
        \State return H
    \EndIf
    \State x.a = k
    \State y = p[x]
    \If{$y = NULL and x.a < y.a$}
        \State CUT(H,x,y)
        \State CASCADING-CUT(H,y)
    \EndIf
    \If{$x.a < H.a$}
        \State H = x
    \algstore{myalg}
\end{algorithmic}
\end{algorithm}

\begin{algorithm}                     
\begin{algorithmic} [1]
\algrestore{myalg}
    \EndIf
    \State return H
    \end{algorithmic}
  \end{algorithm}

The cut function removes x from child list of y and insert in the root list and marks false.\\
The cascading cut function start cutting from y to the root till it get a false marked node. It marks the node true.\\
The worst case cost will be log(n) but the amortized cost is O(1).

\item \textbf{Analysis of whole program}\\
If the block size is m, no. of blocks in a list is m1 and the no. of lists is k, the number of times the functions are running :\\
insert = k*m1*m\\
union = k*m1\\
extract min = k*m1*m\\
decrease key = m1\\
Thus, the amortized cost of the whole program = O(k*m1*m*log(n)) = O(n*log(n))

\end{enumerate}

\end{document}
