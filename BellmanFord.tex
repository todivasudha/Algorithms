\documentclass[a4paper,11pt]{article}
\usepackage[utf8]{inputenc}
\usepackage{times}
\usepackage{algorithm}
\usepackage{algpseudocode}

\title{Report for assignment 10}
\author{Vasudha Todi (14EC10059)}

\begin{document}

\maketitle

\paragraph{Currency conversion using Bellman Ford}
\begin{enumerate}
 \item \textbf{The algorithm}
Bellman Ford is an efficient algorithm for the problem of finding opportunity for profit in currency conversion.\\
  \begin{algorithm}
    \caption{bellmanFord algorithm}
    \begin{algorithmic}[1]
    \State{bellmanFord}{$(X[30][30],s,n)$}
    \For{$i = 0; i < n; i++$}
        \If{$i = s$}
            \State $d[i] = 0$
        \Else
            \State $d[i] = INTMAX$
        \EndIf
    \EndFor
    \For{$i = 0; i < n-1; i++$}
        \For{$u = 0; u < n; u++$}
            \For{$v = 0; v < n; v++$}
                \If{$d[v] > d[u] + X[u][v]$}
                    \State $d[v] = d[u] + X[u][v]$
                \EndIf
            \EndFor
        \EndFor
    \EndFor
    \For{$u = 0; u < n; u++$}
        \For{$v = 0; v < n; v++$}
            \If{$d[v] > d[u] + X[u][v]$}
                \State return 1
            \EndIf
        \EndFor
    \EndFor
    \State return 0
    \end{algorithmic}
  \end{algorithm}
Here -log(X[u][v]/100) is passed into the function.\\

\item \textbf{Analysis of algorithm}\\
The algorithm runs in O(VE) time, where V is the number of vertices and E is the number of edges.\\
$E = n^2$\\
$V = n$\\
Thus, $T(n) = O(n^3)$\\
The algorithm runs just for one source and checks for a negative weight cycle. Since, the graph is fully connected there is no need to check for other sources.\\

\end{enumerate}

\end{document}
