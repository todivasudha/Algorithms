\documentclass[a4paper,11pt]{article}
\usepackage[utf8]{inputenc}
\usepackage{times}
\usepackage{algorithm}
\usepackage{algpseudocode}

\title{Report for assignment 8}
\author{Vasudha Todi (14EC10059)}

\begin{document}

\maketitle


\paragraph{Collision using heaps}
\begin{enumerate}
 \item \textbf{Simulation process for the heap}
Five balls are simulated such that there trajectories can be plotted. Here the collisions have been taken as elastic collisions in 2 dimensions.

  \begin{algorithm}
    \caption{Simulation process}
    \begin{algorithmic}[1]
    \State{Process}{$(T,n)$}
    \If{$t < maxTime$}
        \State e = extractMinHeap(E[], N)
        \If{status of event is invalid}
            \State Process()
            \State return
        \EndIf
        \State Identify whether a 2 particle collision or collision with a wall
        \State Advance the corresponding particles to the collision time
        \State Write in respective files
        \State The respective element lists are traversed and future elements deleted and their status marked as invalid
        \State All possible new events added
        \State Process()
        \State return
    \EndIf
    \State return
    \end{algorithmic}
  \end{algorithm}

\item \textbf{Choice of data types}\\
Two structures are mainly used: one for the state of the ball and other for the events of collision\\
struct state\\
\{\\
        int color;\\
        int radius;\\
        struct point pos;\\
        struct point vel;\\
        float t;\\
        struct node *list;\\
\}\\

struct event\\
\{\\
       int i;\\
        int j;\\
        int status;\\
        float time;\\
        int type;\\
\}\\
The program uses a heap of event pointers as its data structure.\\

 \item \textbf{Complexity analysis for heaps}\\
The complexity of the various heap functions are : \\
Heapify : \\
Maximum time taken is the height of the tree.\\
$T(n) = O(log{n})$\\
Building a heap :\\
Seems to be nlogn but actually it occurs in linear time, i.e., \\
$T(n) = O(n)$\\
Extract minimum :\\
Since this is a minimum heap the topmost element is the minimum element. So time:\\
$T(n) = O(logn)$\\

\end{enumerate}

\end{document}
